\documentclass{article}
\usepackage{graphicx} % Required for inserting images
\usepackage{titlesec}
\setcounter{secnumdepth}{4}

\usepackage{geometry}
\geometry{legalpaper, margin=0.8in}

\renewcommand{\baselinestretch}{1.5} 

\titleformat{\paragraph}
{\normalfont\normalsize\bfseries}{\theparagraph}{1em}{}
\titlespacing*{\paragraph}
{0pt}{3.25ex plus 1ex minus .2ex}{1.5ex plus .2ex}

\title{Rapport M2 Rémi Legrand}
\author{Rémi Legrand}
\date{May 2024}

\begin{document}

\maketitle

\section{Rapide résumé  (futur abstract)}

En se basant sur une méthode développée par Lisa Nickvert et Stéphane dray, il est possible de déterminer les traits lattent des espèces à partir d'un réseau d'interactions grâce à une méthode statistique (AFC/ Correspondance Analysis) qui permet de faire fi des abondances (effet neutre/ champ moyen) et de décomposer le réseau en une série d'axes orthogonaux qui organisent l'information. C'est à dire les traits latents.
Cette méthode, basée sur la décomposition en valeur singulières (SVD) est pratique vu qu'elle permet d'obtenir des axes indépendants et de retirer l'effet des abondance grâce à une "normalisation avec $\chi^2$.

Cependant, bien que cela fonctionne sur des réseaux uniques, est-il possible de faire la même chose sur des séries de réseaux et est ce que la performance augmenterais vu que l'on a une description plus fine et globale du réseau ou est ce que elle diminue vu que l'on pourrait avoir une variabilité trop importante entre les sites et il ne serais plus possible de trouver les axes organisateur de l'ensemble des données mais de chaque sous jeu de données.
Et serait il possible de détecter un des termes récurent dans les analyse de réseaux écologique sur le quel il n'y a pas de consensus: le rewiring?

Rewiring: la modification des interactions entre espèces présentes dans plusieurs sites

En effet, nous appliquons l'AFC sur les tableaux agrégés afin d'obtenir les axes organisateurs du réseau puis après, nous nous attendons à ce ces modification d'interaction entraîne des changements de positons quand on projette un sous-réseau sur les axes obtenus et ainsi que les espèce qui ont une variation de position importante soient des espèce qui ont des "rôles" différents entre les différents sous réseaux et ainsi qu'elles aient des modification de leurs interactions entre plusieurs sites.

Afin de bien controller et de connaitre les paramètre sous jacent, les données sont simulée à partir d'abondances et de probabilité d'interaction entre espèce selon le  

\section{Introduction}

\begin{itemize}
    \item relier au climate change avec les notions de shift de niches
    \item define niche de grinnel, elton (hutchinson) realized and fundamental
    \begin{itemize}
        \item \underline{Grinellian niche:} the niche of the species is determined by the habitat in which it lives and its accompanying behavioural modification. It can be defined by abiotic variables and environmental conditions on broad scale.
        \item \underline{Eltonian niche:} "The niche of an animal means its place in the biotic environment, its relations to food and enemies"
        \item classified niches according to foraging activities.
        \item response to and effect on the environment
        \item \underline{Hutchinsonian niche:} is a n-dimensional hyper-volume
    \end{itemize}
    \item rewiring
    \item trait matching
        \item what is trait matching
        \item how do we quantify it?
    \item introduce what we are going to do
\end{itemize}










\section{Methods}

\subsection{Network notations}
Let's consider a location and the resulting interactions of this sampling. We consider a bipartite network with r resources species and c consumer species. In this case, the resources and consumer doesn't imply a prédation. It can either be plant/pollinators, host/parasitism, or else as long as the network is bipartite. 

Let $Y$ be the $r\times c$ matrix of interactions such that $\textrm{Y} = [y_{ij}]$. With $y_{ij}$ the number interactions between i and j observed.
Let $\textrm{P}=[\frac{y_{ij}}{y_{++}}]$

$r=\textrm{P1}_r =[p_{1+}, ..., p_{r+}]^\intercal$

$c=\textrm{P1}_c=[p_{+1}, ..., p_{+c}]^\intercal$

$y_{++}=\sum_{i=1}^{r}\sum_{j=1}^{c}y_{ij}$

$p_{i+}=\sum_{j=1}^{c}$ and $p_{+j}=\sum_{i=1}^{r}$ are the marginal sums for rows and columns.














\subsection{Simulation}

based on Fründ et al 2016, Benadi et al 2022, Dray and Legrendre 2008 as well as the thesis of Lisa Nickvert

\subsubsection{Assumptions}
Here, we assume that the proxi of the relative abundances is is the frequency of the observed interaction for a given specie. We as well assume that the interactions are solely based on a mixed effect of the trait matching and of the abundance. We can be understood as the probability of two species encountering, which is the product of the abundances, time the probability of these two species interaction together based on their trait matching.

Also, using a bipartite network assume that there is no interaction intra-group but only inter-groups. It implies that the competition, facilitation, spatial exclusion.




\subsubsection{Trait Matching}
Let $\textrm{T}^c = [t^c_{jk}]$ $(c \times 2)$ contains the parameters for the distribution of the consumers' traits. 
Let $\textrm{T}^r = [t^r_{jk}]$ $(r \times 2)$ contains the parameters for the distribution of the resources' traits. 

The first mean of the trait is uniformly distributed between 0 and 1 for both consumers and resources and for the second trait, the span of the gradient is smaller such that it has less weight in driving the matching. 
For the consumer we associate a variance to the distribution in addition to an optimum in order to give a notion of specialisation of the consumers.

The trait matching doesn't have to be something well defined like the length of the beak on the size of the fruit. First of all, it is most likely that the traits found are a combination of multiples pondered traits that can include more subjective thing as the taste for example that can be though of as a gradient of sour/sweetness of the fruit or plant.
The variability/ tolerance is encoded as the variance and the optimum is the mean of the distribution.

The traits PDF follows a normal distribution whose means are uniformly distributed between 0 and 1 and the variance follow a normal distribution.




\subsubsection{Interaction probability based on trait matching}
Then we compute the interaction probability  solely based on the trait matching.


\subsubsection{Abundances and environmental distribution}

Here we use the Hutchinsonian definition of the niche as a position of the species in a 1 dimensional gradient.

Here, we assume that the species are normally distributed across an environmental gradient. This gradient can either represent the distribution over time or the distribution across space, for example in the case of a mountain. As for the traits, this gradients may takes in account different factors such as the humidity, sun exposure, temperature, altitude, soil, etc in the case of the space gradient.
Here we assume normality distribution, which imply a unique associated optimum as well as a tolerance expressed as a variance.

\subsubsection{Interaction probability based on population size}

\subsubsection{Integration of the trait matching and mean field}

\subsubsection{Sampling of the interactions based on the previous computed probability}













\subsection{SVD and correspondance analysis}

\subsubsection{What is an SVD}

The regular diagonalization of a rectangular matrix $A$ of size $n \times m$ by $X^{-1} AX$ the resulting eigenvectors in X will not always be orthogonal, it requires to have a matrix $A$ a square matrix to have $Ax = \lambda x$. Also we have not always enough eigenvalues (apparently, but I don't really get what this means and why).

The singular value decomposition methods enable to diagonalize a rectangular matrix $A$ of size $n \times m$ in two sets of singular vectors, u's and v's. 
$u's \in \mathbb{R}^m$ and $u's \in \mathbb{R}^n$ will be in a $m \times m$ U matrix and a $n \times n$ V matrix

$A = U \Sigma V^\intercal$
$\Leftrightarrow A = \sum_i \sigma_i U_i \otimes V_i^\intercal$
the  $\Sigma$ matrix contains only values on the diagonal and contains the $\sigma$ values that are analogous to the eigenvalues 
We call U the left singular vector and V the right singular vector of $\sigma$
the diagonal entries $\sigma_i = \Sigma_{ii}$ of $\Sigma$ are uniquely determined by $M$ and known as singular values of $M$. The number non 0 singular values is equal to the rank of the matrix $M$.

\subsubsection{Interpretation of the meaning of each matrices of the SVD}

$V^\intercal$ corresponds to a rotation on the unit disc, $\Sigma$ corresponds to a  scaling on the different dimensions, and $U$ to an other rotation 

\subsubsection{Why we are interested in SVD}

SVD enables to obtain a sum of the vector product of rank one matrices, enabling to approximate more and more accurately the origin matrix. Therefore, when we have a matrix made of a Sum of clear signal and noise, we can filter out the noise by selecting the biggest $\sigma$ values as it will constitue the ordered information of the matrix and then there should be something something analog to an eigengap and the rest of the small $\sigma$ are noise.

In the case where the organized information is a sum of products of vectors, it would even be possible to retrieve theses vectors up to a rotation, as long as these are orthogonal.

Also, the regular SVD considers 0 as information whereas in our case, it is rather an absence of information, and that's one of the main reasons for using CA

\subsubsection{Correspondence Analysis}

\paragraph{1: Contribution to $\chi^2$}

contingency table transformed into a contribution table by a poisson $\chi^2$ 
$\chi^2 = \frac{O_{ij} -  E_{ij}}{\sqrt{E_{ij}}} = \sqrt{f_{++}}[\frac{p_{ij} - p_{i+}p_{+j}}{\sqrt{p_{i+}p_{+j}}}]$

$\overline{Q} = [\overline{q}_{ij}] = [\frac{p_{ij} - p_{i+}p_{+j}}{\sqrt{p_{i+}p_{+j}}}] = \frac{\chi_{ij}}{\sqrt{f_{++}}}$

\paragraph{SVD}

Singular value decomposition on the matrix $\overline{Q} = U_{(r\times c)} \Sigma_{(diagonal, c\times c)} V_{(c \times c)}^T$ 
$U$ and $V$ are both column orthogonal matrices and $\Sigma$ is a diagonal matrix $D_{\sigma_i}$ with $\sigma_i \in \mathbb{R}^+$ which are the singular values of $\overline{Q}$


\paragraph{eigen analysis}

\paragraph{comments (to remove later}
Conceptually similar to Principal Component Analysis, but applies to categorical data rather than continuous data

CA has to be applied to a contingency matrix whereas SVD can be applied to any matrix

\subsection{Analysis}

\subsubsection{packages used}

\subsubsection{Trait matching reconstruction principle}

\subsubsection{Evaluate reconstruction}

\subsubsection{Rewiring estimation}

\section{Discussion}

\section{Appendices}

\section{Package}

\section{Ecology}

\end{document}
