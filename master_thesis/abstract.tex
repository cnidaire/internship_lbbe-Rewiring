\section{Abstract}

Climate change is rapidly altering environmental conditions, leading to increases in global average temperatures, drought, and flood events which induce shifts in the species' temporal and spatial niche and impact the global biodiversity. 
To this day, the factors enabling species to adapt and change their interaction partners remain unclear. This lack of understanding may have a significant impact on the future of ecosystems. To better characterize the underlying factors responsible for the adaptation capacity of species, this study investigates whether the estimation of the latent traits of species from a network using Correspondence Analysis, a method developed by Lisa Nicvert, can be extended to quantify interaction rewiring in changing environmental conditions from a series of simulated networks using Foucart Correspondence Analysis (Foucart CA).
We found that Foucart CA can reconstruct the latent traits for a series of networks along an environmental gradient. However, we found no correlation between the current way of computing species' contribution to rewiring nor effect of the environmental and trait tolerance on the rewiring capacity. This suggests that while Foucart CA positions species based on their interaction profiles, it fails to detect profile modifications when changing partners have similar traits. Consequently, although the current method does not satisfyingly compute the individual interaction rewiring, Foucart CA is not a suitable alternative for studying individual interaction rewiring. More methodological research needs to be done to better characterize rewiring.