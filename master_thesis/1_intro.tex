\subsubsection{Introduction}

\paragraph{Introduction to ecology and networks}
Ecology is a field of biology that is focused on the study of species interactions among themselves and with their environment. A way to represent these interactions is by using the network. 

\paragraph{Importance of understanding species interactions in the case of climate change}
In the context of climate change, it is necessary to gain a better understanding of the species and of their plasticity and adaptative capability. This adaptative ability is called rewiring and is the capacity of a species to change its interaction profile. It is an active field of research in ecology as it could enable to better predict the response of the interactions to climate change (\cite{schleuning_trait-based_2020} and \cite{benadi_specialization_2014}) and urbanization (\cite{marcacci_urbanization_2023}) as well as study the resilience of an ecosystem to extinction (\cite{vizentin-bugoni_rewire_2023}). However, the driving factors impacting this capability are yet to be known.


\paragraph{Factors influencing species interactions}
Different from random networks
Different factors are thought to have an impact on the prediction of interactions in bipartite networks \cite{benadi_quantitative_2022}. Species abundance is thought to play a role in the probability of encounter and hence of interaction(\cite{poisot_beyond_2015} and Dormann 2017\cite{dormann_identifying_2017}).
Trait matching is also thought to have an important impact as species with compatible traits are more likely to interact together (species interact when complementary traits: for example body size): Elton 1927 and Ings 2009)
Spatiotemporal distribution, define the phenology and repartition zone, as species need to be in the same location and time to encounter and interact: \cite{caradonna_seeing_2021} and \cite{tylianakis_ecological_2017}.


\paragraph{Niche concepts and their relevance}
Presentation of the environmental and trait niche (for trait matching): 
\begin{itemize}
    \item \underline{Grinellian niche:} \cite{grinnell_geography_1924}The niche of the species is determined by the habitat in which it lives and its accompanying behavioral modification. It can be defined by abiotic variables and environmental conditions on a broad scale.
    \item \underline{Eltonian niche:} \cite{elton_animal_2001} "The niche of an animal means its place in the biotic environment, its relations to food and enemies"
    \item \underline{Hitchinson niche :} \cite{hutchinson_concluding_1957} n-dimensional hypervolume to approximate niche space with Gaussian distribution on each axis (species can be generalist on one axis and specialist on another one)
\end{itemize}


\paragraph{Realized vs fundamental niche}
\begin{itemize}
    \item we care about realized niche no the fundamental one
    \item response to and effect on the environment
\end{itemize}


\paragraph{CA and trait reconstruction}

\paragraph{Structure of the report and what we will see}
