\section{Introduction}


Climate change is rapidly altering environmental conditions, leading to increases in global average temperatures, drought, and flood events \citep{thornton_climate_2014}. These modifications induce shifts in the species' temporal and spatial niche and heavily impact global biodiversity\citep{bellard2012impacts}.

To address this, we need a better understanding of the factors enabling species to adapt and change their interaction partners. Many factors have been hypothesized to drive interactions in ecosystems \citep{benadi_quantitative_2022}. Species abundance is thought to influence the probability of encounter, and consequently, interaction \citep{poisot_beyond_2015, dormann_identifying_2017}. Trait matching is also considered important, as species with compatible traits are more likely to interact (e.g., body size: \cite{benadi_quantitative_2022}). Additionally, spatiotemporal distribution, which defines the phenology and distribution zones, is crucial as species need to be in the same location and time to encounter and interact \citep{tylianakis_ecological_2017, caradonna_seeing_2021}.
A metric to quantify the adaptive capacity of species, called individual contribution to interaction rewiring (individual rewiring), has already been defined by \cite{toju_interaction_2024} and is based on the beta diversity of interactions due to the turnover of interactions at the network scale \citep{poisot_dissimilarity_2012}. However this metric poorly estimates the effect of abundance when computing the turnover of interactions at the network scale.
Therefore, we aim to extend Lisa Nicvert's method \citep{these_lisa_2024}, which uses Correspondence Analysis to estimate the latent traits of the species (related to the interaction profile) and efficiently discards the effect of the abundance, to quantify interaction rewiring in changing environmental conditions from a series of networks.

First, we determine if we can reconstruct the latent traits from a series of networks, using Foucart analysis \citep{foucart_sur_1978}, a multi-network extension of the Correspondence Analysis, using a series of simulated networks across an environmental gradient.
Then we will test if the Foucart Correspondence Analysis can correctly represent variation in species' interaction profile and assess whether environmental and trait tolerance impact the rewiring capability also using a series of simulated networks.
We expect that a change in species' interaction profile will induce changes in their position in Foucart Correspondence Analysis. Additionally, we anticipate that individual interaction rewiring is linked to the environmental tolerance of the niche and the trait tolerance as these factors enable the generalist species to thrive in diverse environmental conditions and interact with a wide range of species.