\section{Conclusion and discussion}


The initial problem was: can Lisa Nicvert's method \citep{these_lisa_2024}, which uses Correspondence Analysis to estimate the latent traits of the species (related to the interaction profile) and efficiently discards the effect of the abundance, be extended to quantify interaction rewiring in changing environmental conditions from a series of networks using Foucart Correspondence Analysis (Foucart CA).

As predicted, we found that Foucart CA is capable of reconstructing latent traits in the multiple network case. However, we did not find a correlation with the individual $\Delta_{OS}$ as defined by \cite{toju_interaction_2024}, nor did we find an effect of the environmental and trait tolerance on the rewiring capability (except for very low environmental tolerances). 
This can be explained by the fact that while Foucart CA positions a species based on its interaction profile, it is unable to detect modifications in the interaction profile when changing partners have similar traits. Hence the absence of the effect of environmental and trait tolerance on the Foucart CA position's variance can not be transposed to the rewiring capacity.

Therefore, Foucart CA is not a suitable method for studying individual interaction rewiring. 
Nonetheless, the individual contribution to rewiring defined by \cite{toju_interaction_2024} has several limitations and should be reconsidered for many reasons. It requires heavy computation when applied to large networks. Additionally, the sum of the individual contributions to rewiring does not equate to the global link beta diversity due to interaction turnover. Finally, it is based on the definition given by \cite{poisot_dissimilarity_2012} that considers species turnover only in terms of presence and absence, thus poorly estimating the abundance effect and leading to an overestimation of rewiring by partially attributing the abundance effects.

To conclude, new methods should be developed to better understand rewiring, properly taking into account the abundance effects and trait matching. I suggest that we could better account for the abundance effect when computing the global link beta diversity due to interaction turnover by assuming that the number of interactions observed for a species serves as a proxy of the abundance of this species, as done in Correspondence Analysis (\ref{abund_effect_ca}).

