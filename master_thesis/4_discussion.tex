\section{Discussion}

The absence of correlation with the method of Toju et al is not a finality on its own, as the way of computing it is not so clean and the current gold standard metric is the beta diversity for the turnover of the interactions. However, the way of computing the $\beta_{rewiring}$ diversity is based on the fact that some of the species are present in one frame and not in the other. Therefore it only takes into account the species disappearance and ignores the fluctuations in abundances that do not lead to the absence of interactions. Hence many species are accounted for in species turnover, greatly underestimating the portion of the beta diversity due to rewiring. Likewise, we think that the proportion due to rewiring (turnover of the interactions) is overestimated.

This is the easiest proxy for the abundances to obtain. However, with the AFC, we can estimate the relative abundances assuming that the abundance for one species is proportional to the number of interactions of this species.

The downsides are that first, this hypothesis may overestimate/underestimate the abundances of the species depending of if these are in an extremity of the environmental gradient.

One of the other downsides is that we lost the information on the proportion of the beta dissimilarity due to rewiring and species turnover. 

One of our hypotheses was that the rewiring as we think about it right now is only an effect of the abundances variation as well as a strong sampling bias.


Not really realistic that only depends on the abundances and the trait matching because even if we assume that we are at equilibrium on a local scale, taking the same network and removing one species would imply that the depending ones will starve. Whereas it would most likely just make more interactions with the others and this is not taken into account and this is what I would call rewiring.