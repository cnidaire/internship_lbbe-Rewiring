\section{Conclusion and discussion}

As expected, Figure \ref{fig:ninter} demonstrates that an increasing number of interactions enables a better reconstruction of the traits, as it reduces the stochasticity of the sampling effect.


Examining the joint influence depicted in figure \ref{fig:frame_env}, we observe that the reconstruction is better for low environmental tolerance values when there is only one sampling location. This is because a single sampled location is placed in the middle of the environmental gradient. With two locations, they are positioned at the two extremities of the gradient. When more locations are included, they are evenly spaced with the first one at the beginning and the last one at the end of the gradient. 
Since all optima are included in the range of the environmental gradient ($0<\mu_{tol\_env}<1$), placing the sampled location in the middle of the gradient results in most of the species being somewhat represented. However, with two locations, some species tend to be missed in each network, leading to nearly disjoint sets of species in the networks. This disjointedness is reinforced by the sampling effect getting stronger (as the total number of observations is fixed), making it challenging for the Foucart CA to find common axes for both networks and explain the drop in trait recovery.

However, increasing the number of frames leads to intermediate networks and a finer description of all the species across the environmental gradient which explains the better performance of trait recovery. Similarly, a higher mean environmental tolerance, results in a more uniform the species abundances across the different sampling locations, thus enhancing the trait recovery.


In Figure \ref{fig:delta}, an increasing weight given to trait matching initially reduces the stochasticity caused by the mean-field effect of the abundances. However, the trait recovery decreases beyond a certain point. This decline might be a decreasing diversity of the species in the interaction profile, i.e., only the present and with the best trait matching will be observed. Thus it will be hard to find common axes.


In Figure \ref{fig:trait_ratio}, we observe that the recovery of trait 1 with axis 1 decreases as the recovery of trait 2 with axis 2 increases, although the recovery for trait 2 does not reach the same level as the first trait. This discrepancy arises because axis one is a composite that holds information about both traits. The overall recovery of the trait information remains constant, however, as we increase the weight given to the second trait, axis 1 increasingly reflects information about the second trait while the second axis becomes more representative of the first trait.



In Figure \ref{fig:tol_trait} we observe that trait recovery decreases as trait tolerance increases.
This may be explained by the fact that the sharper peak of the optimum of the traits indicates more specialized species, making it easier to accurately locate the trait optimum. However, with higher trait tolerance, the probability of slightly miss-positioning the trait optimum increases. Such mispositioning can make minor switches in the trait order and thus significantly reduce the correlation.




As said previously, the current gold standard for computing the rewiring is the beta link diversity due to the turnover of links ($\beta_{OS}$). However, in this case, $\beta_{OS}$ is considered to encompass all the beta link diversity that is not attributable to the turnover of the species. However, the definition of rewiring as defined by $\beta_{OS}$ is very restrictive as it is considered that there is some interactions interaction turnover due to species turnover only when some species interacting in one network are not observed in the second one. Since the variations in abundance are not considered, this method likely overestimates $\beta_{OS}$, while underestimating species turnover.

Therefore, the absence of correlation is not a finality and is not conclusive on its own. Using Correspondence Analysis (CA), we account for species turnover by assuming that species abundance is proportional to the associated number of interactions observed in the network. This approach would offer a better understanding of beta diversity and interaction turnover.




% Downside of this network generation: Not really realistic that only depends on the abundances and the trait matching because even if we assume that we are at equilibrium on a local scale, taking the same network and removing one species would imply that the depending ones will starve. Whereas it would most likely just make more interactions with others and this is not taken into account and this is what I would call rewiring.








However, one of the main hypotheses was that the rewiring depends on the generalism of the species and the witness of the environmental niche does not seem to hold.

expliquer que ce que l'on reconstruit c'est bien les traits et pas les profils d'interactions et que donc on n'observa pas ce que l'on voulais voir, ce qui explique pourquoi est ce queue l'on ne voit pas de effect de la tolérance des traits. Cependant on peut voir qu'il y a bien un effet de l'environement.




In the figures \ref{fig:effect_env} and \ref{fig:effect_trait}, it appears that except when the environmental tolerance is extremely low, there is no correlation between the environmental tolerance and the trait tolerance. This is surprising as we expected that the species capable of rewiring to have a broad environmental niche and to be generalists.

It could be explained by the fact that the CA is reconstructing the trait and we expected it to be analogous to the interaction profile of the species. Hence, it can be considered that the methods work too well to reconstruct the trait matching and hence there is not enough variation in the position when it is interacting with other species with similar traits hence we do not see this shift and it remains stable.

Hence, the Foucart Ca is working well to recover the latent traits of the species in the case of a series of networks, however, it can not be extended to study the rewiring capability of the species in the network. 


